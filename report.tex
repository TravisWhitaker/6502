\documentclass[journal]{IEEEtran}

\usepackage{amsmath}
\usepackage{graphicx}

\begin{document}

\title{MOS Technology 6502 \\ Architecture, Design, and Impact}

\author{Chris~Ranc,~Travis~Whitaker}

\markboth{Journal of Computer Architecture Project Reports, May 2016}
{Journal of Computer Architecture Project Reports, May 2016}

\maketitle

\begin{abstract}
This report discusses the design and architecture of the MOS Technology 6502
CPU. The design of this CPU represents several key breakthroughs in
microprocessor design, and as a result its introduction significantly disrupted
extant markets and created entirely new applicational areas for microprocessor
products. The effects of this CPU's introduction are still observable today.
\end{abstract}

\begin{IEEEkeywords}
MOS Technology, Microprocessor, Computer Architecture
\end{IEEEkeywords}

\IEEEpeerreviewmaketitle

\section{Introduction}

\IEEEPARstart{T}{his} is the indroduction to the paper.

The 6502 is neat.

\section{History}

\subsection{Beginnings at Motorola}

\subsection{Move to MOS Technology}

\subsection{Beginnings of the Microcomputer}

\subsection{Home Game Consoles Emerge}

\subsection{Penetration in the Japanese Market}

\section{Implementation}

\subsection{Features and Specifications}

\subsection{Typical System Configuration}

\section{Programming}

\subsection{Instruction Set}

\subsection{Addressing Modes}

\subsection{Timing Considerations}

\section{Conclusion}
The conclusion goes here.

\begin{thebibliography}{1}

\bibitem{IEEEhowto:kopka}
H.~Kopka and P.~W. Daly, \emph{A Guide to \LaTeX}, 3rd~ed.\hskip 1em plus
  0.5em minus 0.4em\relax Harlow, England: Addison-Wesley, 1999.

\end{thebibliography}

% biography section
% 
% If you have an EPS/PDF photo (graphicx package needed) extra braces are
% needed around the contents of the optional argument to biography to prevent
% the LaTeX parser from getting confused when it sees the complicated
% \includegraphics command within an optional argument. (You could create
% your own custom macro containing the \includegraphics command to make things
% simpler here.)
%\begin{IEEEbiography}[{\includegraphics[width=1in,height=1.25in,clip,keepaspectratio]{mshell}}]{Michael Shell}
% or if you just want to reserve a space for a photo:

\begin{IEEEbiographynophoto}{Chris Ranc}
Something about Chris.
\end{IEEEbiographynophoto}

\begin{IEEEbiographynophoto}{Travis Whitaker}
Something about Travis.
\end{IEEEbiographynophoto}

\end{document}
