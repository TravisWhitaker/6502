\documentclass[journal]{IEEEtran}

\usepackage{amsmath}
\usepackage{graphicx}

\begin{document}

\title{MOS Technology 6502 \\ Architecture, Design, and Impact}

\author{Chris~Ranc,~Travis~Whitaker}

\markboth{Journal of Computer Architecture Project Reports, May 2016}
{Journal of Computer Architecture Project Reports, May 2016}

\maketitle

\begin{abstract}
This report discusses the design and architecture of the MOS Technology 6502
CPU. The design of this CPU represents several key breakthroughs in
microprocessor design, and as a result its introduction significantly disrupted
extant markets and created entirely new applicational areas for microprocessor
products. The effects of this CPU's introduction are still observable today.
\end{abstract}

\begin{IEEEkeywords}
MOS Technology, Microprocessor, Computer Architecture
\end{IEEEkeywords}

\IEEEpeerreviewmaketitle

\section{Introduction}

\IEEEPARstart{T}{he} MOS Technology 6502, typically refered to as simply
``the sixty-five-oh-two,'' is an 8-bit microprocessor introduced in 1975. The
6502's design incorporated several architectural breakthroughs in microprocessor
design, allowing units to be sold at a small fraction of the cost of its
contemporary competitors, including the Motorola 6800, Intel 8080, and Zilog Z80
microprocessors. The 6502's low cost and power requirements are owed to its
small die size. Use of a statically scheduled instruction pipeline and simple
yet novel debugging features enabled the design to approach or even exceed the
throughput of its competitors while utilizing a fraction of the silicon. The low
unit cost and availability of cheap development boards for the 6502 spurred on
the home computer market in the early 1980's, and enabled the design of
affordable home video game consoles.


\section{History}

\subsection{Beginnings at Motorola}

\subsection{Move to MOS Technology}

\subsection{Beginnings of the Microcomputer}

\subsection{Home Game Consoles Emerge}

\subsection{Penetration in the Japanese Market}

\section{Implementation}

\subsection{Features and Specifications}

The 6502 is equipped with three 8-bit general purpose registers: an accumulator
\texttt{A}, and two index registers \texttt{X} and \texttt{Y}. The CPU also has
an 8-bit stack pointer a 16-bit program counter, and an 8-bit processor status
register. The 6502 utilizes a 16-bit memory address bus, however, none of the
general purpose registers available on the CPU are wide enough to accomodate a
whole memory address; this has interesting consequences for programming,
discussed in section IV.

Two interupt generating signals are provided, a maskable level sensitive
interrupt and a non-maskable edge sensitive interrupt. The non-maskable proved
critical for the design of game consoles, which must stop game logic during the
rendering of a frame for display on a television. Due to the limited memory
available on 6502-based systems, many game consoles would compute and buffer a
single scan line of the output image at once, rather than buffering the entire
frame. This rendering technique required only a small amount of memory, but
total use of the CPU throughout the frame rendering process was necessary. On
many 6502-based game consoles, the non-maskable interrupt was simply tied to the
60 Hz (America) or 50 Hz (Japan, Europe) vertical blanking synchronization
signal.

The original 6502 die used an 8 $\mu m$ process on a 3.9 x 4.3 mm die area,
embedded in a 40-pin ceramic DIP unit. Approximately 15\% of the area on the
original die was dedicated to the instruction decoding PLA. However, the PLA was
only mask-programmable, and there was effectively no way to update the
instruction decoding logic on the original 6502 chips. This would prove to be
problematic, as rushed development led to several quirks in the original
manufacturing run. The first very first lot of units shipped in 1975 exhibited
a bug in the \texttt{ROR} instruction, causing it to behave as a \texttt{LSR}
instruction without carry, and a bug in the \texttt{JMP} instruction when
performing an indirect jump. Several other batches and clones from other
manufacturers are known for various other strange quirks or undocumented
opcodes.

One of the most touted features of the Motorola 6800, the 6502's principal
competitor, was its extensive debugging facilities. The 6800 could be easily
halted with a non-maskable debugging interrupt, and the hardware even supported
something of a primitive core dump feature without the need for dual-port RAM.
The 6502 features only one built in debugging feature in contrast: a bus ready
signal. Exposing the bus ready signal was a simple enhancement, but proved
invaluable when debugging 6502 programs. A program under test could effectively
be ``stepped-through'' in hardware, by overriding the system clock with a
``step-through'' signal that may only be asserted when the bus ready signal
indicated the CPU was ready to move on to the next instruction. The 6502 was the
first affordable microprocessor with development boards featuring such a
hardware step-through feature, and this lone debugging facility proved
sufficient for the majority of 6502 program development.

\subsection{Typical System Configuration}

Contemporary 6502-based systems often features 1-2 MHz clock signals. Although
this is a small fraction of the clock frequency often used in systems based on
competing CPUs (e.g. the Motorola 68A70, a 6800 derivative, could operate at up
to 10 MHz), the 6502 pipeline allowed for similar throughputs to be achieved.
The control unit utilized both the rising edge and falling edge of the clock
signal for pipeline synchronization, allowing for partial overlap between
instruction fetch and execution.

The 6502 has a hard-wired memory map, so there is little variation in memory
architecture among 6502-based systems. Addresses 0100-01FF are reserved for the
stack, 4000-7FFF are reserved for memory-mapped IO components, and all addresses
FFF9 and up are reserved for the interrupt vector tables. Most early 6502-based
systems such as the Apple II, Commodore PET, and the Atari 2600 reserved
addresses 8000-FFF9 for program ROM. However, a technique known as bank
switching, first used on the Nintendo Famicom/NES, allowed using a portion of
this address space for indexing into one of several smaller memory banks. Care
had to be taken to reserve a critical code region for duplication on all banks,
otherwise program continuity would be impossible. This technique allowed
Famicom/NES games to approach several megabits in size; the largest licensed
game was \emph{Kirby's Adventure} with a 6 MBit ROM.

The memory-mapped IO region was critical for 6502-based home computers and game
consoles. Home computers and terminal-based development boards often used this
region for disk drives, tape drives, or printers. It was possible to use this
region for drawing video; the Commodore PET's video terminal operated this way.
However, only a small region of the framebuffer could be written out at a time,
so video terminals implemented with this technique often suffered from extremely
slow frame refresh rates. This was a concious trade-off made by Commodore for
the PET, as the system was designed strictly for the business market. Other
systems marketed for gaming, such as the Famicom/NES and the Commodore 64, used
dedicated video units with direct memory access to allow for a full frame to be
buffered for each television vertical blank synchronization. This freed CPU time
during rendering as well, as the CPU only had to initiate a DMA transfer for
each line of the frame. More advanced video effects could be accomplished, such
as dithering and transparency, by executing other memory operations during frame
drawing.

\section{Programming}

\subsection{Instruction Set}

\subsection{Addressing Modes}

\subsection{Interrupts and Timing Considerations}

The 6502 features two programmable interrupts: an edge-triggered non-maskable
interrupt, and a level-triggered maskable interrupt. The maskable interrupt
is triggered by a high voltage on an open collector, or by execution of the
\emph{BRK} instruction, and jumps to address FFFE in the interrupt vector table.
The non-maskable interrupt is triggered by pulling the NMI line low. This
behavior may be leveraged by external components to detect when the NMI handler
is finished executing; the NMI is non-nestable, so the NMI signal won't be
pulled high again until the interrupt handler has returned. The NMI causes the
cpu to jump to address FFFA in the interrupt vector table.

Upon interrupt receipt, the CPU finishes the current instruction. Then the
program counter and program status registers are pushed onto the stack. The
program counter is then loaded with the address in the appropriate entry in the
interrupt vector table. The interrupt handler returns by executing the
\texttt{RTI} return from interrupt instruction. There are only 256 bytes
available on the 6502's stack, and each interrupt handler stack frame occupies
three such bytes. Therefore, it is critical that no deeply nested interrupt
handlers are executed on the 6502.

The 6502's reset signal also acts somewhat like a non-programmable interrupt.
After CPU start up is complete, the program counter is set to the address at
FFFC in the interrupt vector table. Typically this vector entry will contain
the address of the first byte of program ROM.

\section{Conclusion}
The conclusion goes here.

\begin{thebibliography}{1}

\bibitem{IEEEhowto:kopka}
H.~Kopka and P.~W. Daly, \emph{A Guide to \LaTeX}, 3rd~ed.\hskip 1em plus
  0.5em minus 0.4em\relax Harlow, England: Addison-Wesley, 1999.

\end{thebibliography}

\begin{IEEEbiographynophoto}{Chris Ranc}
Something about Chris.
\end{IEEEbiographynophoto}

\begin{IEEEbiographynophoto}{Travis Whitaker}
Something about Travis.
\end{IEEEbiographynophoto}

\end{document}
